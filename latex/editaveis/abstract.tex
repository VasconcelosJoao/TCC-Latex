\begin{resumo}[Abstract]
  \begin{otherlanguage*}{english}
    This work presents the development of an educational software in Portuguese, aimed at Linux and Windows operating systems, to facilitate the learning of basic SQL structures for Brazilian students. This project addresses the challenges faced by students in understanding SQL due to linguistic barriers related to the predominance of English in technical terminology. The methodology adopted includes a qualitative approach, complemented by the use of the agile methodology Extreme Programming (XP), focusing on simplicity, modularity, and scalability. The software enables the translation of SQL commands from Portuguese to ANSI SQL standard, with support for DDL and DML commands, and provides an intuitive interface for executing, translating, and analyzing commands.The expected results include reducing linguistic barriers in SQL learning, promoting academic inclusion, and providing an accessible interface for beginners. The modular architecture ensures efficient integration with MySQL databases, as well as the potential for expansion to new languages and features. In conclusion, the proposed tool can significantly contribute to programming education by offering a practical and user-friendly environment that improves understanding and academic performance in the use of SQL.
    
    \vspace{\onelineskip}
  
    \noindent 
    \textbf{Key-words}: SQL. Educational software. Command translation. Programming learning.
  \end{otherlanguage*}
 \end{resumo}
 