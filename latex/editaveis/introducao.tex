
\chapter[Introdução]{Introdução}
\label{sec:Introducao}

\section{Contextualização}

Os bancos de dados relacionais exercem um papel imprescindível na sociedade sendo responsáveis por persistir e organizar diversas informações de diferentes áreas e domínios.

Diante disso, uma importante ferramenta para contribuir com os estudo de bancos de dados é a linguagem SQL (\textit{Structured Query Language}), sendo fundamental para a manipulação e gerenciamento de bancos de dados.

Garantir que bancos de dados sejam bem projetados depende de pessoas capacitadas. Com o uso de dados em diversas áreas, a compreensão do SQL torna-se essencial para estudantes de engenharia de software e áreas afins. No entanto, muitos estudantes enfrentam dificuldades no aprendizado dessa linguagem devido à falta de conhecimento do inglês e por consequência às estruturas básicas do SQL.

Para isso é necessário que o estudante tenha conhecimentos prévios sobre inglês e consiga entender as palavras chaves utilizadas na linguagem SQL, assim permitindo que considere usa-las em problemas. Conforme identificado, os participantes traduziam as palavras-chave para sua língua nativa na tentativa de resolver um problema, o que frequentemente resultava em erros na questão \cite{Miedema2021}.

Portanto observar a linguagem nativa torna-se um fator importante na construção do aprendizado do aluno, propiciando que alcancem melhores resultados acadêmicos através de ferramentas em sua língua nativa \cite{Silva2020}.

Sendo assim, compreendendo as barreiras enfrentadas por estudantes brasileiros na aprendizagem do SQL, é interessante fornecer uma ferramenta para computador com o intuito de fornecer uma estrutura básica em português, que facilite a entrada do estudante brasileiros no aprendizado de SQL, sem que seja imposta barreiras linguísticas a essa pessoa, e condicionando o aprendizado adequado.


\section{Questão da Pesquisa}

Para a elaboração do trabalho mostra-se relevante compreender como auxiliar estudantes a aprenderem SQL, diante disso a seguinte questão de pesquisa foi proposta:

Como o desenvolvimento de um software educativo pode auxiliar estudantes no aprendizado das estruturas básicas do SQL?

\section{Justificativa}
A proposta deste trabalho justifica-se pela necessidade de ferramentas que tornem o aprendizado do SQL mais acessível e interativo. O software desenvolvido visa preencher uma lacuna existente na educação em SQL, especialmente para falantes da língua portuguesa, proporcionando um ambiente amigável e intuitivo para iniciantes.

Em contraponto aos modelos de text-to-SQL, que promovem o uso de texto em linguagem natural para a realização de consultas SQL por meio de modelos treinados, que traduzem o texto para SQL e gream consultas \cite{Jose2023}, propõe-se a criação de um ambiente em português para desenvolvimento voltado para indivíduos com dificuldades no domínio do inglês, com o objetivo de facilitar o aprendizado das estruturas básicas do SQL.

% podemos Falar sobre a importancia do uso de computadores no aprendizado de programacao

% Falar sobre acessibilidade

\section{Objetivos}
\label{sec:objetivos}

\subsection{Objetivo Geral}

O Objetivo principal do trabalho é desenvolver um software educativo em português para o sistema operacional Linux e Windows, fornecendo um ambiente simples e intuitivo, além do acesso as estruturas básicas do SQL.

\subsection{Objetivos específicos}
\begin{itemize}
    \item Desenvolver um software que permita a conversão de comandos em português para SQL.
    \item Facilitar o aprendizado das estruturas básicas do SQL, incluindo DDL e DML.
    \item Avaliar a eficácia do software na melhoria da compreensão dos alunos sobre SQL.
\end{itemize}

\section{Metodologia}

 As metodologias utilizadas para a realização deste trabalho, foram divididas em duas categorias: Metodologia de Pesquisa e Metodologia de Desenvolvimento de Software.

\subsection{Metodologia de pesquisa}

Sobre a Metodologia de pesquisa, será aplicado a classificação proposta por pelo autor \cite{Gerhardt2009}, que descreve diferentes tipos de pesquisa em relação a abordagem, natureza e procedimentos, enquanto ao conceito de produção tecnológica segue o definido por \cite{Serzedello_Tomael_2011} como um procedimento de pesquisa.

\begin{table}[ht]
    \centering
    
    \begin{tabular}{|c|c|c|c|}
    \hline
    \textbf{Abordagem} & \textbf{Natureza} & \textbf{Objetivos} & \textbf{Procedimentos} \\ \hline
    Qualitativa        & Aplicada          & Exploratória       & Pesquisa bibliográfica \\ 
                    &                   &                    & Produção tecnológica   \\ \hline
    \end{tabular}
    \caption{Metodologia de pesquisa}
    Fonte: Autoria própria.
    
\end{table}

Aspectos que envolvem à metodologia de pesquisa em questão serão melhor abordados na seção ~\ref{sec:metodologia}.

\subsection{Metodologia de desenvolvimento de software}

A metodologia de desenvolvimento que será utilizada neste trabalho segue valores princípios e práticas do  \emph{Extreme Programming} \cite{beckKent2004} sendo estes:


\begin{itemize}
   \item Valores:
   \begin{itemize}
       \item Simplicidade
       \item Coragem
    \end{itemize}
    \item Princípios:
    \begin{itemize}
       \item Qualidade
       \item \textit{Baby steps}
    \end{itemize}
    \item Práticas:
    \begin{itemize}
       \item \textit{Weekly cicle}
       \item \textit{Incremental Design}
    \end{itemize}
\end{itemize}

 
Aspectos que envolvem à metodologia de desenvolvimento de software em questão serão melhor abordados na seção ~\ref{sec:metodologia_de_desenvolvimento_de_software}.

\section{Organização do trabalho}

Para a organização do trabalho forma feitas as seguintes divisões:

\begin{itemize}
   \item Capítulo ~\ref{sec:Introducao} - Introdução: Refere-se ao capitulo que descreve o projeto proposto, bem como a questão de pesquisa, justificativa, objetivo geral e específicos e uma breve introdução da metodologia.
   \item Capítulo ~\ref{sec:referencial} - Referencial Teórico: indica o capítulo de  discursão, que indica o entendimento do tema bem como nuances que contribuem para o trabalho.
   \item Capítulo ~\ref{sec:metodologia} - Metodologia : compreende aspectos voltados as metodologias do trabalho, contando essencialmente por metodologia de pesquisa e metodologia de desenvolvimento de software.
   \item Capítulo ~\ref{sec:considerações} - Conclusão : Encerra a primeira fase do trabalho, bem como resume o plano de execução para o trabalho proposto.
\end{itemize}