\chapter[Considerações]{Considerações Finais}
\label{sec:considerações}


O presente trabalho tem como objetivo principal, conforme descrito na seção \ref{sec:objetivos}, a criação de um software educativo em português, compatível com os sistemas operacionais Linux e Windows, que ofereça um ambiente acessível e intuitivo para o aprendizado das estruturas básicas da linguagem SQL. Para tal, propõe-se o desenvolvimento de uma ferramenta que viabilize a conversão de comandos em português para SQL, facilitando a compreensão e o domínio de conceitos fundamentais, como DDL (Data Definition Language) e DML (Data Manipulation Language).

Este trabalho investiga, por meio de uma pesquisa bibliográfica, bancos de dados relacionais e a linguagem SQL. O estudo detalha as linguagens de definição (DDL) e manipulação de dados (DML) de SQL, explicando comandos como CREATE, ALTER, DROP, TRUNCATE, RENAME, INSERT, UPDATE e DELETE. Por fim, discute-se o impacto das barreiras linguísticas no aprendizado de SQL, destacando como ferramentas educacionais em língua nativa podem mitigar essas dificuldades, facilitando o processo de assimilação técnica dos estudantes.

A proposta deste trabalho fundamenta-se na metodologia adotada, caracterizando-se por uma abordagem qualitativa, de natureza aplicada, com objetivo exploratório, incluindo pesquisa bibliográfica e produção tecnológica. Quanto à metodologia de desenvolvimento de software, baseia-se em valores, princípios e práticas do Extreme Programming (XP).

A arquitetura do sistema foi projetada com foco na organização, garantindo uma estrutura modular e bem definida que promove a clareza e a manutenção do projeto ao longo do tempo. A divisão lógica dos componentes assegura que cada parte desempenhe um papel específico, facilitando tanto a escalabilidade quanto a integração de novos recursos. Esse cuidado com a organização reflete-se na simplificação dos fluxos internos e na capacidade de identificar e corrigir rapidamente quaisquer inconsistências ou necessidades de melhoria, consolidando uma base sólida para a continuidade e evolução do sistema.

Portanto, a proposta apresentada visa proporcionar aos estudantes brasileiros uma ferramenta educativa que facilite o aprendizado das estruturas básicas do SQL em sua língua nativa. Tal abordagem pode reduzir as barreiras linguísticas enfrentadas por esses estudantes, promovendo uma compreensão mais efetiva e acessível da linguagem SQL.