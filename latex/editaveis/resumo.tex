\begin{resumo}
    O presente trabalho apresenta o desenvolvimento de um software educativo em português, direcionado aos sistemas operacionais Linux e Windows, com o objetivo de facilitar o aprendizado das estruturas básicas de SQL para estudantes brasileiros. Este projeto surge em resposta às dificuldades enfrentadas por alunos na compreensão da linguagem SQL, devido a barreiras linguísticas associadas à predominância do inglês na terminologia técnica. A metodologia de pesquisa inclui uma abordagem qualitativa, complementada pelo uso da metodologia de desenvolvimento ágil Extreme Programming (XP), focando na simplicidade, modularidade e escalabilidade do sistema. O software permite a tradução de comandos SQL em português para o padrão ANSI SQL, com suporte a comandos DDL e DML, e oferece uma interface intuitiva para execução, tradução e análise dos comandos. Os resultados esperados incluem a redução das barreiras linguísticas no aprendizado do SQL, maior inclusão acadêmica e uma interface acessível para iniciantes. A arquitetura modular do sistema assegura a integração eficiente com bancos de dados MySQL, bem como a possibilidade de expansão para novos idiomas e funcionalidades. Conclui-se que a ferramenta proposta pode contribuir significativamente para o aprendizado de programação, oferecendo um ambiente prático e amigável, que melhora o entendimento e o desempenho acadêmico dos estudantes no uso da linguagem SQL.
    
    \vspace{\onelineskip}
    
    \noindent
    \textbf{Palavras-chave}: SQL. Software educativo. Tradução de comandos. Aprendizado de programação.
    \end{resumo}
